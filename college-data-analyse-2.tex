\documentclass{beamer}

% \documentclass[draft]{beamer}
% \includeonlyframes{wip}

% This file is a solution template for:

% - Giving a talk on some subject.
% - The talk is between 15min and 45min long.
% - Style is ornate.



\mode<presentation>
{
  \usetheme{Pittsburgh}
  \setbeamertemplate{navigation symbols}{}
  % or ...

  % \setbeamercovered{transparent}
  % or whatever (possibly just delete it)

  \setbeamertemplate{footline}
  {
    \begin{beamercolorbox}[wd=\paperwidth,right,sep=1ex]{beamer color}
      \insertframenumber
    \end{beamercolorbox}
  }
}


\usepackage[dutch]{babel}
\usepackage[utf8]{inputenc}
\usepackage[T1]{fontenc}

\usepackage{mathtools}
\usepackage{tikz}

\usetikzlibrary{arrows,external}
\usepackage{pgfplots}
\pgfplotsset{compat=1.10}
\usepackage[detect-family,output-decimal-marker={,}]{siunitx}
% \usepackage[eulergreek]{sansmath}
% \sisetup{text-sf=\sansmath}
\usepackage{relsize}
\newenvironment{sansmath}{}{}


\graphicspath{{figures/}}


\renewcommand{\leadsto}{\ \rightarrow\ }
\newcommand{\tsum}{\textstyle\sum}
\newcommand{\rchisq}{\tilde\chi^2}


\title{Statistische data-analyse 2}
\subtitle{Kleinste kwadraten en de $\chi^2$-test}
\author{Natuurkundepracticum}
\institute{
  Vrije Universiteit / Universiteit van Amsterdam
}
\date{}

\subject{Colleges}
% This is only inserted into the PDF information catalog. Can be left
% out.



% If you have a file called "university-logo-filename.xxx", where xxx
% is a graphic format that can be processed by latex or pdflatex,
% resp., then you can add a logo as follows:

% \pgfdeclareimage[height=0.5cm]{university-logo}{university-logo-filename}
% \logo{\pgfuseimage{university-logo}}



% Delete this, if you do not want the table of contents to pop up at
% the beginning of each subsection:
\AtBeginSubsection[]
{
  \begin{frame}<beamer>{Outline}
    \tableofcontents[currentsection,currentsubsection]
  \end{frame}
}


% If you wish to uncover everything in a step-wise fashion, uncomment
% the following command:

%\beamerdefaultoverlayspecification{<+->}


\begin{document}

\begin{frame}
  \titlepage
\end{frame}

\begin{frame}{Outline}
  \tableofcontents
  % You might wish to add the option [pausesections]
\end{frame}


% Since this a solution template for a generic talk, very little can
% be said about how it should be structured. However, the talk length
% of between 15min and 45min and the theme suggest that you stick to
% the following rules:

% - Exactly two or three sections (other than the summary).
% - At *most* three subsections per section.
% - Talk about 30s to 2min per frame. So there should be between about
%   15 and 30 frames, all told.

\section{Kleinste kwadratenmethode}

\begin{frame}{Fitten van data}
  \begin{itemize}
    \item Je doet nooit zomaar een serie metingen
    \item Je hebt meestal een bepaalde verwachting op basis van een theoretisch model
    \pause
    \item Voorbeeld (volume van een gas bij verschillende temperaturen, bij constante druk):
    \begin{equation*}
      V = \mathrm{const}\cdot T + V_0,
    \end{equation*}
    met $V$ het volume, $T$ de temperatuur van het gas in \si{\celsius}, en $V_0$ het volume van het gas bij \SI{0}{\celsius}.
  \end{itemize}
\end{frame}

\begin{frame}{Probleemstelling}
  Aangenomen dat het verband
  \begin{equation*}
    y = A + Bx
  \end{equation*}
  geldt, wat zijn dan de meest waarschijnlijke waardes van de parameters $A$ en $B$ gegeven een serie metingen $(x_i, y_i)$.
\end{frame}

\begin{frame}{Schatten}
  \begin{center}
    \begin{tikzpicture}
      % axis
      \draw[thick,->] (0, 0) -- (6, 0) node[right] {$x$};
      \draw[thick,->] (0, 0) -- (0, 4) node[above] {$y$};
      \only<1-2>{\input{scripts/slide-fitten-van-lijn-1.out}}
      \draw<2>[red] (0, 1) -- (6, 3.4);
      \only<3-4>{\input{scripts/slide-fitten-van-lijn-2.out}}
      \draw<4-5>[red] (0, 1) -- (6, 3.4) node[right] {?};
      \only<5>{\input{scripts/slide-fitten-van-lijn-3.out}}
    \end{tikzpicture}

    \only<1>{\phantom{Makkelijk!}}
    \only<2>{Makkelijk!}
    \only<3-4>{Niet zo makkelijk...}
    \only<5>{Voor welke lijn zijn de afwijkingen minimaal?}
  \end{center}
\end{frame}

\begin{frame}{Lineaire Regressie}
  \begin{center}
    \begin{tikzpicture}[scale=.6]
      % axis
      \draw[thick,->] (0, 0) -- (6, 0) node[right] {$x$};
      \draw[thick,->] (0, 0) -- (0, 4) node[above] {$y$};
      \draw[red] (0, 1) -- (6, 3.4) node[right] {?};
      \input{scripts/slide-fitten-van-lijn-3.out}
      \path (4.66666666667, 2.1762050576) -- (4.66666666667, 2.86666666667) node[midway, right] {$\Delta y_i$};
    \end{tikzpicture}
  \end{center}
  \uncover<2->{
  Minimaliseer $g=\sum\Delta y_i^2 = \sum(y_i - A - Bx_i)^2$:
  \begin{gather*}
    \uncover<3->{\frac{\partial g}{\partial A} = 2\sum(y_i - A - Bx_i) = 0} \uncover<4->{\leadsto \alert{\textstyle\sum y_i - AN - B\textstyle\sum x_i = 0}} \\
    \uncover<5->{\frac{\partial g}{\partial B} = 2\sum x_i(y_i - A - Bx_i) = 0}
    \uncover<6->{\leadsto \alert{\textstyle\sum x_i y_i - A\textstyle\sum x_i - B\textstyle\sum x_i^2 = 0}}
  \end{gather*}
  \uncover<7->{Twee vergelijkingen, twee onbekenden.}}
\end{frame}

\begin{frame}{Lineaire Regressie}
  Oplossing:
  \begin{gather*}
    A = \frac{\tsum x^2\tsum y - \tsum x\tsum xy}{N\tsum x^2 - (\tsum x)^2} \\[1em]
    B = \frac{N\tsum xy - \tsum x\tsum y}{N\tsum x^2 - (\tsum x)^2}
  \end{gather*}
  Ziet er misschien wat intimiderend uit, maar is eigenlijk eenvoudig uit te rekenen. En dan weet je de \emph{beste} lijn!
\end{frame}

\begin{frame}{Controle: Schatting Fout}
  \begin{center}
    \begin{tikzpicture}[scale=.6]
      % axis
      \draw[thick,->] (0, 0) -- (6, 0) node[right] {$x$};
      \draw[thick,->] (0, 0) -- (0, 4) node[above] {$y$};
      \draw[red] (0, 1) -- (6, 3.4);
      \input{scripts/slide-fitten-van-lijn-3.out}
      \path (4.66666666667, 2.1762050576) -- (4.66666666667, 2.86666666667) node[midway, right] {$\Delta y_i$};
    \end{tikzpicture}
  \end{center}
  Als het verwachte verband inderdaad geldt, dan vormt de lijn de \alert{werkelijke waarden}, en zijn de metingen normaal verdeeld rondom die waardes. De afwijkingen $\Delta y_i = y_i - A - Bx_i$ zouden dan dus normaal verdeeld moeten zijn. We kunnen de standaarddeviatie dan vergelijken met de door ons geschatte meetfouten. Klopt dat? Fijn!
  \uncover<2->{
  \begin{tikzpicture}[remember picture, overlay]
    \draw[fill=white,opacity=.7] (current page.south west) rectangle (current page.north east);
    \node [rotate=10,scale=2,red,text width=.6\linewidth,align=center]
    at (current page.center) {ALLEEN als onzekerheden gelijk zijn voor alle punten.};
  \end{tikzpicture}}
\end{frame}

\begin{frame}{Controle: Schatting Fout}
  Gemiddelde:
  \begin{equation*}
    \bar x = \frac{1}{N}\sum x_i  \tag{0 parameters}
  \end{equation*}
  Standaarddeviatie normale verdeling:
  \begin{equation*}
    \sigma = \sqrt{\frac{1}{N-1}\sum(x_i - \bar x)^2} \tag{1 parameter}
  \end{equation*}
  Schatting meetonzekerheid:
  \begin{equation*}
    \sigma_y = \sqrt{\frac{1}{N-2}\sum(y_i - A - Bx_i)^2} \tag{2 parameters}
  \end{equation*}
  \pause
  De factor $N - p$, met $p$ het aantal geschatte parameters, wordt nog belangrijk!
\end{frame}

\begin{frame}{Onzekerheden op $A$ en $B$}
  \dots zijn direct uit te rekenen met behulp van foutenberekening uit jaar 1. Je neemt aan dat de fout op $x$ gelijk is aan nul. Je neemt alléén de fouten op $y$ mee:
  \begin{gather*}
    \sigma_A = \sigma_y\sqrt{\frac{\tsum x^2}{N\tsum x^2 - (\tsum x)^2}} \\[1em]
    \sigma_B = \sigma_y\sqrt{\frac{N}{N\tsum x^2 - (\tsum x)^2}}
  \end{gather*}
  \uncover<2->{
  \begin{tikzpicture}[remember picture, overlay]
    \draw[fill=white,opacity=.7] (current page.south west) rectangle (current page.north east);
    \node [rotate=10,scale=2,red,text width=.6\linewidth,align=center]
    at (current page.center) {ALLEEN als onzekerheden gelijk zijn voor alle punten.};
  \end{tikzpicture}}
\end{frame}

\begin{frame}{Fouten zijn belangrijk}
  \begin{center}
    \begin{tikzpicture}
      % axis
      \draw[thick,->] (0, 0) -- (6, 0) node[right] {$x$};
      \draw[thick,->] (0, 0) -- (0, 4) node[above] {$y$};
      \only<1>{\input{scripts/slide-fitten-van-lijn-4.out}}
      \only<2>{\input{scripts/slide-fitten-van-lijn-5.out}}
      \draw[red] (0, 1) -- (6, 3.4);
    \end{tikzpicture}

    \only<1>{Rechte lijn? Dat kan!}
    \only<2>{Rechte lijn? Waarschijnlijk niet!}
  \end{center}
\end{frame}

\begin{frame}{Vollediger: Normale Verdeling}
  Minimaliseer \alert{niet}:
  \begin{equation*}
    g = \sum(y_i - A - Bx_i)^2,
  \end{equation*}
  maar
  \begin{equation*}
    h = \sum\frac{(y_i - A - Bx_i)^2}{{\sigma_{y_i}}^2}.
  \end{equation*}
  Bij gelijke onzekerheden geldt
  \begin{equation*}
    h = \frac{1}{{\sigma_y}^2}\sum(y_i - A - Bx_i)^2,
  \end{equation*}
  en dat valt er uit bij gelijkstellen aan nul (zoals we tot nu toe gedaan hebben). Bij verschillende $\sigma_{y_i}$ moet je het netjes doen.
\end{frame}

\begin{frame}{Méér dan lijnen alleen}
  Stelsel vergelijkingen met $A, B, \ldots$ is altijd oplosbaar als de vergelijkingen \alert{lineair} zijn. Hier kun je aan fitten:
  \begin{equation*}
    y = A + Bx + Cx^2 + Dx^3 + \ldots
  \end{equation*}
  \pause
  En ook:
  \begin{equation*}
    y = A\sin x + B\cos x
  \end{equation*}
  \pause
  Maar niet:
  \begin{equation*}
    y = A\cdot e^{Bx}
  \end{equation*}
  \pause
  Daar staat tegenover dat je wel weer kunt \alert{lineariseren}:
  \begin{equation*}
    \ln y = \ln A + Bx  \qquad (y' = A' + Bx)
  \end{equation*}
\end{frame}

\begin{frame}{De computer kan alles}
  De computer is goed in dom rekenwerk, en kan verschillende waardes voor $A$ en $B$ blijven proberen, net zolang totdat de \emph{som van kwadraten} $h$ klein genoeg is! Dus óók functies die niet te fitten zijn met lineaire regressie (bv. $y = A\cdot e^{Bx}$) kun je zo alsnog fitten. Er zijn veel slimme algoritmes in omloop om snel de beste waardes te vinden.
\end{frame}


\section{$\chi^2$-test}

\begin{frame}{Hoe goed is de fit?}
  Maar hoe weten we hoe \emph{goed} de fit is? We minimaliseerden net
  \begin{equation*}
    h = \sum\frac{(y_i - A - Bx_i)^2}{{\sigma_{y_i}}^2} \equiv \chi^2
  \end{equation*}
  ofwel
  \begin{equation*}
    \chi^2 = \sum\left(\frac{\text{meetwaarde} - \text{verwachte waarde}}{\text{standaarddeviatie}}\right)^2
  \end{equation*}
  Hoe kleiner deze waarde is, hoe beter! Maar wat is een realistische waarde van $\chi^2$?
\end{frame}

\begin{frame}{Van $\chi^2$ naar $\rchisq$}
  We hadden
  \begin{equation*}
    \sigma_y = \sqrt{\frac{1}{N-2}\sum(y_i - A - Bx_i)^2}
  \end{equation*}
  Dan is
  \begin{equation*}
    \chi^2 = \frac{\sum(y_i - A - Bx_i)^2}{{\sigma_y}^2} = N - 2
  \end{equation*}
  dus afhankelijk van het aantal meetpunten $N$. Handig is nu de \alert{gereduceerde $\chi^2$} als
  \begin{equation*}
    \rchisq = \frac{\chi^2}{N - 2} = 1
  \end{equation*}
  Als alles klopt hebben we dus een $\rchisq$-waarde van 1 gevonden in onze fit.
\end{frame}

\begin{frame}{Kanttekening}
  \begin{itemize}
    \item Om de kwaliteit van een fit te schatten moet je de standaarddeviatie niet achteraf uitrekenen (je gaat er dan namelijk al van uit dat je hypothese klopt en het verband geldig is)
    \pause
    \item Schat de standaarddeviatie vóóraf op basis van verwachte meetonzekerheden
    \pause
    \item Klopt je hypothese? Dan vind je $\rchisq \approx 1$
  \end{itemize}
\end{frame}

\begin{frame}{$\chi^2$-test procedure}
  \begin{itemize}
    \item Schat meetonzekerheden $\sigma_{y_i}$
    \item Doe een fit aan de verwachte functie $f(x)$
    \item Bereken
    \begin{equation*}
      \chi^2 = \sum\left(\frac{y_i - f(x_i)}{\sigma_{y_i}}\right)^2
    \end{equation*}
    \item Bereken
    \begin{equation*}
      \rchisq = \frac{\chi^2}{N - p},
    \end{equation*}
    met $p$ het aantal parameters in de functie $f(x)$
  \end{itemize}
\end{frame}

\begin{frame}{$\chi^2$-test}
  \begin{itemize}
    \item $\rchisq \approx 1$: hypothese zeer waarschijnlijk juist
    \item $\rchisq \gg 1$: hypothese zeer waarschijnlijk onjuist
    \item $\rchisq \ll 1$: de fit valt wel héél mooi over de datapunten (+ onzekerheid!). De hypothese is juist (maar dan is de onzekerheid te groot geschat) óf een andere functie past misschien óók, en dan weet je dus niet welke er klopt...
  \end{itemize}
\end{frame}

\begin{frame}{Van kwaliteit naar kwantitatief}
  Leuk zo'n test, maar de opmerking dat $\rchisq \approx 1$ is wel behoorlijk vaag. Kunnen we dat kwantificeren? Ja, maar dan moeten we kijken naar wat voor $\rchisq$-waardes we verwachten als we het experiment vaak herhalen en de hypothese klopt.
\end{frame}

\begin{frame}{$\chi^2$-verdeling}
  \begin{center}
    \only<1>{% \usepackage{tikz}
% \usetikzlibrary{arrows,external}
% \usepackage{pgfplots}
% \pgfplotsset{compat=1.10}
% \usepackage[detect-family]{siunitx}
% \usepackage[eulergreek]{sansmath}
% \sisetup{text-sf=\sansmath}
% \usepackage{relsize}
%
    \tikzsetnextfilename{externalized-plot-chisq-1.tex}
\pgfkeysifdefined{/artist/width}
    {\pgfkeysgetvalue{/artist/width}{\defaultwidth}}
    {\def\defaultwidth{ .67\linewidth }}
%
%
\begin{sansmath}
\begin{tikzpicture}[
        font=\sffamily,
        every pin/.style={inner sep=2pt, font={\sffamily\smaller}},
        every label/.style={inner sep=2pt, font={\sffamily\smaller}},
        every pin edge/.style={<-, >=stealth', shorten <=2pt},
        pin distance=2.5ex,
    ]
    \begin{axis}[
            axis background/.style={  },
            xmode=normal,
            ymode=normal,
            width=\defaultwidth,
            scale only axis,
            axis equal=false,
            %
            title={  },
            %
            xlabel={ x },
            ylabel={ y },
            %
            xmin={ 0 },
            xmax={ 6 },
            ymin={ 0 },
            ymax={ 5 },
            %
            xtick={  },
            ytick={  },
            xticklabel style={  },
            yticklabel style={  },
            %
            tick align=outside,
            max space between ticks=40,
            every tick/.style={},
            axis on top,
            point meta min={  },
            point meta max={  },
            colormap={coolwarm}{
                rgb255=( 59, 76,192)
                rgb255=( 98,130,234)
                rgb255=(141,176,254)
                rgb255=(184,208,249)
                rgb255=(221,221,221)
                rgb255=(245,196,173)
                rgb255=(244,154,123)
                rgb255=(222, 96, 77)
                rgb255=(180,  4, 38)},
            colormap={whiteblack}{gray=(1) gray=(0)},
            colormap={blackwhite}{gray=(0) gray=(1)},
        ]

        


    
    % Draw series plot
    \addplot[mark=*,mark options=white,only marks] coordinates {
            (0.5, 1.6873036091)
            (1.33333333333, 1.34980640924)
            (2.16666666667, 1.70821514099)
            (3.0, 1.87810941335)
            (3.83333333333, 2.79295562213)
            (4.66666666667, 2.1762050576)
            (5.5, 3.72344352926)
    };
    

    
        % Draw y-error bars
                \draw (axis cs:0.5, 1.3873036091) --
                      (axis cs:0.5, 1.9873036091);
                \draw (axis cs:1.33333333333, 1.04980640924) --
                      (axis cs:1.33333333333, 1.64980640924);
                \draw (axis cs:2.16666666667, 1.40821514099) --
                      (axis cs:2.16666666667, 2.00821514099);
                \draw (axis cs:3.0, 1.57810941335) --
                      (axis cs:3.0, 2.17810941335);
                \draw (axis cs:3.83333333333, 2.49295562213) --
                      (axis cs:3.83333333333, 3.09295562213);
                \draw (axis cs:4.66666666667, 1.8762050576) --
                      (axis cs:4.66666666667, 2.4762050576);
                \draw (axis cs:5.5, 3.42344352926) --
                      (axis cs:5.5, 4.02344352926);
    
    % Draw series plot
    \addplot[mark=o,only marks] coordinates {
            (0.5, 1.6873036091)
            (1.33333333333, 1.34980640924)
            (2.16666666667, 1.70821514099)
            (3.0, 1.87810941335)
            (3.83333333333, 2.79295562213)
            (4.66666666667, 2.1762050576)
            (5.5, 3.72344352926)
    };
    

    
    % Draw series plot
    \addplot[no markers,solid] coordinates {
            (0.0, 1.0)
            (6.0, 3.4)
    };
    

    \node[,
          below left=2pt
        ]
        at (rel axis cs:1,1)
        { $\tilde\chi^2 = 2.7$ };

    \end{axis}
\end{tikzpicture}
\end{sansmath}}

    \only<2>{% \usepackage{tikz}
% \usetikzlibrary{arrows,external}
% \usepackage{pgfplots}
% \pgfplotsset{compat=1.10}
% \usepackage[detect-family]{siunitx}
% \usepackage[eulergreek]{sansmath}
% \sisetup{text-sf=\sansmath}
% \usepackage{relsize}
%
    \tikzsetnextfilename{externalized-plot-chisq-2.tex}
\pgfkeysifdefined{/artist/width}
    {\pgfkeysgetvalue{/artist/width}{\defaultwidth}}
    {\def\defaultwidth{ .67\linewidth }}
%
%
\begin{sansmath}
\begin{tikzpicture}[
        font=\sffamily,
        every pin/.style={inner sep=2pt, font={\sffamily\smaller}},
        every label/.style={inner sep=2pt, font={\sffamily\smaller}},
        every pin edge/.style={<-, >=stealth', shorten <=2pt},
        pin distance=2.5ex,
    ]
    \begin{axis}[
            axis background/.style={  },
            xmode=normal,
            ymode=normal,
            width=\defaultwidth,
            scale only axis,
            axis equal=false,
            %
            title={  },
            %
            xlabel={ x },
            ylabel={ y },
            %
            xmin={ 0 },
            xmax={ 6 },
            ymin={ 0 },
            ymax={ 5 },
            %
            xtick={  },
            ytick={  },
            xticklabel style={  },
            yticklabel style={  },
            %
            tick align=outside,
            max space between ticks=40,
            every tick/.style={},
            axis on top,
            point meta min={  },
            point meta max={  },
            colormap={coolwarm}{
                rgb255=( 59, 76,192)
                rgb255=( 98,130,234)
                rgb255=(141,176,254)
                rgb255=(184,208,249)
                rgb255=(221,221,221)
                rgb255=(245,196,173)
                rgb255=(244,154,123)
                rgb255=(222, 96, 77)
                rgb255=(180,  4, 38)},
            colormap={whiteblack}{gray=(1) gray=(0)},
            colormap={blackwhite}{gray=(0) gray=(1)},
        ]

        


    
    % Draw series plot
    \addplot[mark=*,mark options=white,only marks] coordinates {
            (0.5, 0.971637929731)
            (1.33333333333, 1.62904506215)
            (2.16666666667, 1.79185555402)
            (3.0, 2.63863238111)
            (3.83333333333, 1.91529112048)
            (4.66666666667, 2.76994150546)
            (5.5, 3.0847836936)
    };
    

    
        % Draw y-error bars
                \draw (axis cs:0.5, 0.671637929731) --
                      (axis cs:0.5, 1.27163792973);
                \draw (axis cs:1.33333333333, 1.32904506215) --
                      (axis cs:1.33333333333, 1.92904506215);
                \draw (axis cs:2.16666666667, 1.49185555402) --
                      (axis cs:2.16666666667, 2.09185555402);
                \draw (axis cs:3.0, 2.33863238111) --
                      (axis cs:3.0, 2.93863238111);
                \draw (axis cs:3.83333333333, 1.61529112048) --
                      (axis cs:3.83333333333, 2.21529112048);
                \draw (axis cs:4.66666666667, 2.46994150546) --
                      (axis cs:4.66666666667, 3.06994150546);
                \draw (axis cs:5.5, 2.7847836936) --
                      (axis cs:5.5, 3.3847836936);
    
    % Draw series plot
    \addplot[mark=o,only marks] coordinates {
            (0.5, 0.971637929731)
            (1.33333333333, 1.62904506215)
            (2.16666666667, 1.79185555402)
            (3.0, 2.63863238111)
            (3.83333333333, 1.91529112048)
            (4.66666666667, 2.76994150546)
            (5.5, 3.0847836936)
    };
    

    
    % Draw series plot
    \addplot[no markers,solid] coordinates {
            (0.0, 1.0)
            (6.0, 3.4)
    };
    

    \node[,
          below left=2pt
        ]
        at (rel axis cs:1,1)
        { $\tilde\chi^2 = 1.5$ };

    \end{axis}
\end{tikzpicture}
\end{sansmath}}

    \only<3>{% \usepackage{tikz}
% \usetikzlibrary{arrows,external}
% \usepackage{pgfplots}
% \pgfplotsset{compat=1.10}
% \usepackage[detect-family]{siunitx}
% \usepackage[eulergreek]{sansmath}
% \sisetup{text-sf=\sansmath}
% \usepackage{relsize}
%
    \tikzsetnextfilename{externalized-plot-chisq-3.tex}
\pgfkeysifdefined{/artist/width}
    {\pgfkeysgetvalue{/artist/width}{\defaultwidth}}
    {\def\defaultwidth{ .67\linewidth }}
%
%
\begin{sansmath}
\begin{tikzpicture}[
        font=\sffamily,
        every pin/.style={inner sep=2pt, font={\sffamily\smaller}},
        every label/.style={inner sep=2pt, font={\sffamily\smaller}},
        every pin edge/.style={<-, >=stealth', shorten <=2pt},
        pin distance=2.5ex,
    ]
    \begin{axis}[
            axis background/.style={  },
            xmode=normal,
            ymode=normal,
            width=\defaultwidth,
            scale only axis,
            axis equal=false,
            %
            title={  },
            %
            xlabel={ x },
            ylabel={ y },
            %
            xmin={ 0 },
            xmax={ 6 },
            ymin={ 0 },
            ymax={ 5 },
            %
            xtick={  },
            ytick={  },
            xticklabel style={  },
            yticklabel style={  },
            %
            tick align=outside,
            max space between ticks=40,
            every tick/.style={},
            axis on top,
            point meta min={  },
            point meta max={  },
            colormap={coolwarm}{
                rgb255=( 59, 76,192)
                rgb255=( 98,130,234)
                rgb255=(141,176,254)
                rgb255=(184,208,249)
                rgb255=(221,221,221)
                rgb255=(245,196,173)
                rgb255=(244,154,123)
                rgb255=(222, 96, 77)
                rgb255=(180,  4, 38)},
            colormap={whiteblack}{gray=(1) gray=(0)},
            colormap={blackwhite}{gray=(0) gray=(1)},
        ]

        


    
    % Draw series plot
    \addplot[mark=*,mark options=white,only marks] coordinates {
            (0.5, 1.5401308327)
            (1.33333333333, 1.20336595314)
            (2.16666666667, 1.8149382044)
            (3.0, 1.93664247462)
            (3.83333333333, 2.54599745735)
            (4.66666666667, 3.04151123078)
            (5.5, 2.86981424684)
    };
    

    
        % Draw y-error bars
                \draw (axis cs:0.5, 1.2401308327) --
                      (axis cs:0.5, 1.8401308327);
                \draw (axis cs:1.33333333333, 0.903365953139) --
                      (axis cs:1.33333333333, 1.50336595314);
                \draw (axis cs:2.16666666667, 1.5149382044) --
                      (axis cs:2.16666666667, 2.1149382044);
                \draw (axis cs:3.0, 1.63664247462) --
                      (axis cs:3.0, 2.23664247462);
                \draw (axis cs:3.83333333333, 2.24599745735) --
                      (axis cs:3.83333333333, 2.84599745735);
                \draw (axis cs:4.66666666667, 2.74151123078) --
                      (axis cs:4.66666666667, 3.34151123078);
                \draw (axis cs:5.5, 2.56981424684) --
                      (axis cs:5.5, 3.16981424684);
    
    % Draw series plot
    \addplot[mark=o,only marks] coordinates {
            (0.5, 1.5401308327)
            (1.33333333333, 1.20336595314)
            (2.16666666667, 1.8149382044)
            (3.0, 1.93664247462)
            (3.83333333333, 2.54599745735)
            (4.66666666667, 3.04151123078)
            (5.5, 2.86981424684)
    };
    

    
    % Draw series plot
    \addplot[no markers,solid] coordinates {
            (0.0, 1.0)
            (6.0, 3.4)
    };
    

    \node[,
          below left=2pt
        ]
        at (rel axis cs:1,1)
        { $\tilde\chi^2 = 1.0$ };

    \end{axis}
\end{tikzpicture}
\end{sansmath}}

    \only<4>{% \usepackage{tikz}
% \usetikzlibrary{arrows,external}
% \usepackage{pgfplots}
% \pgfplotsset{compat=1.10}
% \usepackage[detect-family]{siunitx}
% \usepackage[eulergreek]{sansmath}
% \sisetup{text-sf=\sansmath}
% \usepackage{relsize}
%
    \tikzsetnextfilename{externalized-plot-chisq-4.tex}
\pgfkeysifdefined{/artist/width}
    {\pgfkeysgetvalue{/artist/width}{\defaultwidth}}
    {\def\defaultwidth{ .67\linewidth }}
%
%
\begin{sansmath}
\begin{tikzpicture}[
        font=\sffamily,
        every pin/.style={inner sep=2pt, font={\sffamily\smaller}},
        every label/.style={inner sep=2pt, font={\sffamily\smaller}},
        every pin edge/.style={<-, >=stealth', shorten <=2pt},
        pin distance=2.5ex,
    ]
    \begin{axis}[
            axis background/.style={  },
            xmode=normal,
            ymode=normal,
            width=\defaultwidth,
            scale only axis,
            axis equal=false,
            %
            title={  },
            %
            xlabel={ x },
            ylabel={ y },
            %
            xmin={ 0 },
            xmax={ 6 },
            ymin={ 0 },
            ymax={ 5 },
            %
            xtick={  },
            ytick={  },
            xticklabel style={  },
            yticklabel style={  },
            %
            tick align=outside,
            max space between ticks=40,
            every tick/.style={},
            axis on top,
            point meta min={  },
            point meta max={  },
            colormap={coolwarm}{
                rgb255=( 59, 76,192)
                rgb255=( 98,130,234)
                rgb255=(141,176,254)
                rgb255=(184,208,249)
                rgb255=(221,221,221)
                rgb255=(245,196,173)
                rgb255=(244,154,123)
                rgb255=(222, 96, 77)
                rgb255=(180,  4, 38)},
            colormap={whiteblack}{gray=(1) gray=(0)},
            colormap={blackwhite}{gray=(0) gray=(1)},
        ]

        


    
    % Draw series plot
    \addplot[mark=*,mark options=white,only marks] coordinates {
            (0.5, 1.54341711295)
            (1.33333333333, 1.80381054951)
            (2.16666666667, 2.01741496834)
            (3.0, 2.47025678478)
            (3.83333333333, 2.32821497558)
            (4.66666666667, 2.82979959901)
            (5.5, 2.91926916972)
    };
    

    
        % Draw y-error bars
                \draw (axis cs:0.5, 1.24341711295) --
                      (axis cs:0.5, 1.84341711295);
                \draw (axis cs:1.33333333333, 1.50381054951) --
                      (axis cs:1.33333333333, 2.10381054951);
                \draw (axis cs:2.16666666667, 1.71741496834) --
                      (axis cs:2.16666666667, 2.31741496834);
                \draw (axis cs:3.0, 2.17025678478) --
                      (axis cs:3.0, 2.77025678478);
                \draw (axis cs:3.83333333333, 2.02821497558) --
                      (axis cs:3.83333333333, 2.62821497558);
                \draw (axis cs:4.66666666667, 2.52979959901) --
                      (axis cs:4.66666666667, 3.12979959901);
                \draw (axis cs:5.5, 2.61926916972) --
                      (axis cs:5.5, 3.21926916972);
    
    % Draw series plot
    \addplot[mark=o,only marks] coordinates {
            (0.5, 1.54341711295)
            (1.33333333333, 1.80381054951)
            (2.16666666667, 2.01741496834)
            (3.0, 2.47025678478)
            (3.83333333333, 2.32821497558)
            (4.66666666667, 2.82979959901)
            (5.5, 2.91926916972)
    };
    

    
    % Draw series plot
    \addplot[no markers,solid] coordinates {
            (0.0, 1.0)
            (6.0, 3.4)
    };
    

    \node[,
          below left=2pt
        ]
        at (rel axis cs:1,1)
        { $\tilde\chi^2 = 0.9$ };

    \end{axis}
\end{tikzpicture}
\end{sansmath}}

    \only<5>{% \usepackage{tikz}
% \usetikzlibrary{arrows,external}
% \usepackage{pgfplots}
% \pgfplotsset{compat=1.10}
% \usepackage[detect-family]{siunitx}
% \usepackage[eulergreek]{sansmath}
% \sisetup{text-sf=\sansmath}
% \usepackage{relsize}
%
    \tikzsetnextfilename{externalized-plot-chisq-5.tex}
\pgfkeysifdefined{/artist/width}
    {\pgfkeysgetvalue{/artist/width}{\defaultwidth}}
    {\def\defaultwidth{ .67\linewidth }}
%
%
\begin{sansmath}
\begin{tikzpicture}[
        font=\sffamily,
        every pin/.style={inner sep=2pt, font={\sffamily\smaller}},
        every label/.style={inner sep=2pt, font={\sffamily\smaller}},
        every pin edge/.style={<-, >=stealth', shorten <=2pt},
        pin distance=2.5ex,
    ]
    \begin{axis}[
            axis background/.style={  },
            xmode=normal,
            ymode=normal,
            width=\defaultwidth,
            scale only axis,
            axis equal=false,
            %
            title={  },
            %
            xlabel={ x },
            ylabel={ y },
            %
            xmin={ 0 },
            xmax={ 6 },
            ymin={ 0 },
            ymax={ 5 },
            %
            xtick={  },
            ytick={  },
            xticklabel style={  },
            yticklabel style={  },
            %
            tick align=outside,
            max space between ticks=40,
            every tick/.style={},
            axis on top,
            point meta min={  },
            point meta max={  },
            colormap={coolwarm}{
                rgb255=( 59, 76,192)
                rgb255=( 98,130,234)
                rgb255=(141,176,254)
                rgb255=(184,208,249)
                rgb255=(221,221,221)
                rgb255=(245,196,173)
                rgb255=(244,154,123)
                rgb255=(222, 96, 77)
                rgb255=(180,  4, 38)},
            colormap={whiteblack}{gray=(1) gray=(0)},
            colormap={blackwhite}{gray=(0) gray=(1)},
        ]

        


    
    % Draw series plot
    \addplot[mark=*,mark options=white,only marks] coordinates {
            (0.5, 1.11963357611)
            (1.33333333333, 1.69243997335)
            (2.16666666667, 1.65916844115)
            (3.0, 2.08097394194)
            (3.83333333333, 2.3271815233)
            (4.66666666667, 2.61310497422)
            (5.5, 2.99862616075)
    };
    

    
        % Draw y-error bars
                \draw (axis cs:0.5, 0.819633576112) --
                      (axis cs:0.5, 1.41963357611);
                \draw (axis cs:1.33333333333, 1.39243997335) --
                      (axis cs:1.33333333333, 1.99243997335);
                \draw (axis cs:2.16666666667, 1.35916844115) --
                      (axis cs:2.16666666667, 1.95916844115);
                \draw (axis cs:3.0, 1.78097394194) --
                      (axis cs:3.0, 2.38097394194);
                \draw (axis cs:3.83333333333, 2.0271815233) --
                      (axis cs:3.83333333333, 2.6271815233);
                \draw (axis cs:4.66666666667, 2.31310497422) --
                      (axis cs:4.66666666667, 2.91310497422);
                \draw (axis cs:5.5, 2.69862616075) --
                      (axis cs:5.5, 3.29862616075);
    
    % Draw series plot
    \addplot[mark=o,only marks] coordinates {
            (0.5, 1.11963357611)
            (1.33333333333, 1.69243997335)
            (2.16666666667, 1.65916844115)
            (3.0, 2.08097394194)
            (3.83333333333, 2.3271815233)
            (4.66666666667, 2.61310497422)
            (5.5, 2.99862616075)
    };
    

    
    % Draw series plot
    \addplot[no markers,solid] coordinates {
            (0.0, 1.0)
            (6.0, 3.4)
    };
    

    \node[,
          below left=2pt
        ]
        at (rel axis cs:1,1)
        { $\tilde\chi^2 = 0.5$ };

    \end{axis}
\end{tikzpicture}
\end{sansmath}}

  \end{center}
\end{frame}

\begin{frame}{$\chi^2$-verdeling}
  \begin{center}
    \only<1>{\input{scripts/slide-chisq-histogram-N-3}}

    \only<2>{\input{scripts/slide-chisq-histogram-N-5}}

    \only<3>{\input{scripts/slide-chisq-histogram-N-7}}

    \only<4>{\input{scripts/slide-chisq-histogram-N-14}}

  \end{center}
  Oppervlakte onder limietverdeling geeft weer de kans dat de gemeten waarde in dat gebied ligt.
\end{frame}

\begin{frame}{Eenzijdige overschrijdingskans}
  Stel, je vindt $\rchisq = \num{2,0}$, bij een fit aan een lijn (2 parameters) en 10 meetpunten, dus 8 vrijheidsgraden. Dan is de kans maar \SI{4,2}{\percent} dat de hypothese klopt én je minstens zó'n slechte $\rchisq$ vindt. Hmm... geloven we dan eigenlijk wel dat de hypothese klopt?
\end{frame}

\begin{frame}{$\chi^2$-test}
  Bepaal
  \begin{equation*}
    p = \text{Prob}(\rchisq \geq \rchisq_o)
  \end{equation*}
  met $\rchisq_o$ de gemeten $\rchisq$-waarde. Meestal wordt de hypothese verworpen als $p < \num{0,05}$. Je mag zelf kiezen, als je maar de $\rchisq$-waarde vermeldt, liefst met $p$-waarde. Zo kan iedereen je conclusie controleren.
  \pause
  \alert{Zie ook Appendix D in Taylor.}

  \pause
  \alert{Noem de $p$-waarde in je verslag!}
\end{frame}

\begin{frame}{$\chi^2$-test}
  Er wordt wel eens gezegd:
  \begin{quote}
    Als je een $p$-waarde hebt kleiner dan \num{0,05}, dan is de kans dat je de hypothese \alert{ten onrechte} verwerpt kleiner dan \SI{5}{\percent}.
  \end{quote}

  \pause
  Voorbeeld: we denken dat ons verband een rechte lijn is, maar we krijgen een $p$-waarde van \num{0,04}. Dan is de kans dat het een rechte lijn is \alert{en onze metingen kloppen} heel klein. We verwerpen dus deze hypothese. Is de kans dat het tóch een rechte lijn is nu \SI{4}{\percent}?

  \pause
  \begin{tikzpicture}[remember picture, overlay]
    \node [yshift=3em,rotate=20,scale=10,red,text opacity=.5]
    at (current page.center) {NEE!};
  \end{tikzpicture}

  \pause
  (Of de natuur zich op een bepaalde manier gedraagt is een \alert{feit}, geen kans.)
\end{frame}

\begin{frame}{Pas op met fitten!}
  \begin{center}
    \only<1>{\input{scripts/slide-polynoom-fit-O-1}}

    \only<2>{\input{scripts/slide-polynoom-fit-O-2}}

    \only<3>{\input{scripts/slide-polynoom-fit-O-3}}

    \only<4>{\input{scripts/slide-polynoom-fit-O-4}}

    \only<5>{\input{scripts/slide-polynoom-fit-O-5}}

    \only<6-7>{\input{scripts/slide-polynoom-fit-O-6}}

    \only<8>{\input{scripts/slide-polynoom-fit-O-1}}

    \only<9>{\input{scripts/slide-polynoom-fit-O-1-err}}

  \end{center}

  \only<1-6>{\phantom{Dit heeft geen zin\ldots}}
  \only<7>{Dit heeft geen zin\ldots}
  \only<8>{Model klopt niet (helemaal)? Een \emph{resultaat}! Maar\ldots}
  \only<9>{Fouten onderschat? Dit past beter!}
\end{frame}



\section*{Samenvatting}

\begin{frame}{Samenvatting}

  Vandaag behandeld:
  % Keep the summary *very short*.
  \begin{itemize}
  \item
    De kleinste-kwadraten methode om de meest waarschijnlijke parameters in een fit te bepalen
  \item
    Hoe goed de fit is, is te interpreteren door de gereduceerde $\chi^2$ te berekenen. Deze waarde komt gemiddeld uit rond 1, \alert{als de hypothese klopt}
  \item
    Gebruik de $p$-waarde om in te schatten hoe onwaarschijnlijk de gevonden $\rchisq$-waarde is
  \end{itemize}

  \uncover<2->{
  \vskip0pt plus.5fill
  \begin{itemize}
  \item
    Niet behandeld (zie Taylor):
    \begin{itemize}
    \item
      Veel van de wiskundige afleidingen
    \item
      Wanneer mag je foutieve metingen verwerpen?
    \item
      Correlatie, binomiale verdeling en Poissonverdeling
    \end{itemize}}
  \end{itemize}
\end{frame}

\end{document}
