\documentclass{beamer}

% \documentclass[draft]{beamer}
% \includeonlyframes{wip}

% This file is a solution template for:

% - Giving a talk on some subject.
% - The talk is between 15min and 45min long.
% - Style is ornate.



\mode<presentation>
{
  \usetheme{Pittsburgh}
  % or ...

  % \setbeamercovered{transparent}
  % or whatever (possibly just delete it)
}


\usepackage[dutch]{babel}
\usepackage[utf8]{inputenc}
\usepackage{times}
\usepackage[T1]{fontenc}

\usepackage{mathtools}
\usepackage{tikz}

\usetikzlibrary{arrows,external}
\usepackage{pgfplots}
\pgfplotsset{compat=1.10}
\usepackage[detect-family]{siunitx}
\usepackage[eulergreek]{sansmath}
\sisetup{text-sf=\sansmath}
\usepackage{relsize}


\graphicspath{{figures/}}


\renewcommand{\leadsto}{\ \rightarrow\ }
\newcommand{\tsum}{\textstyle\sum}


\title{Statistische data-analyse 2}
\subtitle{Kleinste kwadraten en de $\chi^2$-test}
\author{David Fokkema}
\institute{
  Practicum natuurkunde \\
  Vrije Universiteit / Universiteit van Amsterdam
}
\date{16 januari 2017}

\subject{Colleges}
% This is only inserted into the PDF information catalog. Can be left
% out.



% If you have a file called "university-logo-filename.xxx", where xxx
% is a graphic format that can be processed by latex or pdflatex,
% resp., then you can add a logo as follows:

% \pgfdeclareimage[height=0.5cm]{university-logo}{university-logo-filename}
% \logo{\pgfuseimage{university-logo}}



% Delete this, if you do not want the table of contents to pop up at
% the beginning of each subsection:
\AtBeginSubsection[]
{
  \begin{frame}<beamer>{Outline}
    \tableofcontents[currentsection,currentsubsection]
  \end{frame}
}


% If you wish to uncover everything in a step-wise fashion, uncomment
% the following command:

%\beamerdefaultoverlayspecification{<+->}


\begin{document}

\begin{frame}
  \titlepage
\end{frame}

\begin{frame}{Outline}
  \tableofcontents
  % You might wish to add the option [pausesections]
\end{frame}


% Since this a solution template for a generic talk, very little can
% be said about how it should be structured. However, the talk length
% of between 15min and 45min and the theme suggest that you stick to
% the following rules:

% - Exactly two or three sections (other than the summary).
% - At *most* three subsections per section.
% - Talk about 30s to 2min per frame. So there should be between about
%   15 and 30 frames, all told.

\section{Kleinste kwadratenmethode}

\begin{frame}{Fitten van data}
  \begin{itemize}
    \item Je doet nooit zomaar een serie metingen
    \item Je hebt meestal een bepaalde verwachting op basis van een theoretisch model
    \item Voorbeeld (volume van een gas bij verschillende temperaturen, bij constante druk):
    \begin{equation*}
      V = \mathrm{const}\cdot T + V_0,
    \end{equation*}
    met $V$ het volume, $T$ de temperatuur van het gas in \si{\celsius}, en $V_0$ het volume van het gas bij \SI{0}{\celsius}.
  \end{itemize}
\end{frame}

\begin{frame}{Probleemstelling}
  Aangenomen dat het verband
  \begin{equation*}
    y = A + Bx
  \end{equation*}
  geldt, wat zijn dan de meest waarschijnlijke waardes van de parameters $A$ en $B$ gegeven een serie metingen $(x_i, y_i)$.
\end{frame}

\begin{frame}{Schatten}
  \begin{center}
    \begin{tikzpicture}
      % axis
      \draw[thick,->] (0, 0) -- (6, 0) node[right] {$x$};
      \draw[thick,->] (0, 0) -- (0, 4) node[above] {$y$};
      \only<1-2>{\input{scripts/slide-fitten-van-lijn-1.out}}
      \draw<2>[red] (0, 1) -- (6, 3.4);
      \only<3-4>{\input{scripts/slide-fitten-van-lijn-2.out}}
      \draw<4-5>[red] (0, 1) -- (6, 3.4) node[right] {?};
      \only<5>{\input{scripts/slide-fitten-van-lijn-3.out}}
    \end{tikzpicture}

    \only<1>{\phantom{Makkelijk!}}
    \only<2>{Makkelijk!}
    \only<3-4>{Niet zo makkelijk...}
    \only<5>{Voor welke lijn zijn de afwijkingen minimaal?}
  \end{center}
\end{frame}

\begin{frame}{Lineaire Regressie}
  \begin{center}
    \begin{tikzpicture}[scale=.6]
      % axis
      \draw[thick,->] (0, 0) -- (6, 0) node[right] {$x$};
      \draw[thick,->] (0, 0) -- (0, 4) node[above] {$y$};
      \draw[red] (0, 1) -- (6, 3.4) node[right] {?};
      \input{scripts/slide-fitten-van-lijn-3.out}
      \path (4.66666666667, 2.1762050576) -- (4.66666666667, 2.86666666667) node[midway, right] {$\Delta y_i$};
    \end{tikzpicture}
  \end{center}
  Minimaliseer $g=\sum\Delta y_i^2 = \sum(y_i - A - Bx_i)^2$:
  \begin{gather*}
    \frac{\partial g}{\partial A} = 2\sum(y_i - A _ Bx_i) = 0 \leadsto \alert{\textstyle\sum y_i - AN - B\textstyle\sum x_i = 0}, \\
    \frac{\partial g}{\partial B} = 2\sum x_i(y_i - A - Bx_i) = 0 \leadsto \alert{\textstyle\sum x_i y_i - A\textstyle\sum x_i - B\textstyle\sum x_i^2 = 0}.
  \end{gather*}
  Twee vergelijkingen, twee onbekenden.
\end{frame}

\begin{frame}{Lineaire Regressie}
  Oplossing:
  \begin{gather*}
    A = \frac{\tsum x^2\tsum y - \tsum x\tsum xy}{N\tsum x^2 - (\tsum x)^2}, \\[1em]
    B = \frac{N\tsum xy - \tsum x\tsum y}{N\tsum x^2 - (\tsum x)^2}.
  \end{gather*}
  Ziet er misschien wat intimiderend uit, maar is eigenlijk eenvoudig uit te rekenen. En dan weet je de \emph{beste} lijn!
\end{frame}

\begin{frame}{Controle: Schatting Fout}
  \begin{center}
    \begin{tikzpicture}[scale=.6]
      % axis
      \draw[thick,->] (0, 0) -- (6, 0) node[right] {$x$};
      \draw[thick,->] (0, 0) -- (0, 4) node[above] {$y$};
      \draw[red] (0, 1) -- (6, 3.4);
      \input{scripts/slide-fitten-van-lijn-3.out}
      \path (4.66666666667, 2.1762050576) -- (4.66666666667, 2.86666666667) node[midway, right] {$\Delta y_i$};
    \end{tikzpicture}
  \end{center}
  Als het verwachte verband inderdaad geldt, dan vormt de lijn de \alert{werkelijke waarden}, en zijn de metingen normaal verdeeld rondom die waardes. De afwijkingen $\Delta y_i = y_i - A - Bx_i$ zouden dan dus normaal verdeeld moeten zijn. We kunnen de standaarddeviatie dan vergelijken met de door ons geschatte meetfouten. Klopt dat? Fijn!
\end{frame}

\begin{frame}{Controle: Schatting Fout}
  Gemiddelde:
  \begin{equation*}
    \bar x = \frac{1}{N}\sum x_i  \tag{0 parameters}
  \end{equation*}
  Standaarddeviatie normale verdeling:
  \begin{equation*}
    \sigma = \sqrt{\frac{1}{N-1}\sum(x_i - \bar x)^2} \tag{1 parameter}
  \end{equation*}
  Schatting meetonzekerheid:
  \begin{equation*}
    \sigma_y = \sqrt{\frac{1}{N-2}\sum(y_i - A - Bx_i)^2} \tag{2 parameters}
  \end{equation*}
  \pause
  De factor $N - p$, met $p$ het aantal geschatte parameters, wordt nog belangrijk!
\end{frame}

\begin{frame}{Onzekerheden op $A$ en $B$}
  \dots zijn direct uit te rekenen met behulp van foutenberekening uit jaar 1. Je neemt aan dat de fout op $x$ gelijk is aan nul. Je neemt alléén de fouten op $y$ mee:
  \begin{gather*}
    \sigma_A = \sigma_y\sqrt{\frac{\tsum x^2}{N\tsum x^2 - (\tsum x)^2}}, \\[1em]
    \sigma_B = \sigma_y\sqrt{\frac{N}{N\tsum x^2 - (\tsum x)^2}}.
  \end{gather*}
\end{frame}

\begin{frame}{Vollediger: Normale Verdeling}
  Minimaliseer \alert{niet}:
  \begin{equation*}
    g = \sum(y_i - A - Bx_i)^2,
  \end{equation*}
  maar
  \begin{equation*}
    h = \frac{\sum(y_i - A - Bx_i)^2}{{\sigma_{y_i}}^2}.
  \end{equation*}
  Bij gelijke onzekerheden geldt
  \begin{equation*}
    h = \frac{1}{{\sigma_y}^2}\sum(y_i - A - Bx_i)^2,
  \end{equation*}
  en dat valt er uit bij gelijkstellen aan nul (zoals we tot nu toe gedaan hebben). Bij verschillende $\sigma_{y_i}$ moet je het netjes doen.
\end{frame}

\begin{frame}{Méér dan lijnen alleen}
  Stelsel vergelijkingen met $A, B, \ldots$ is altijd oplosbaar als de vergelijkingen \alert{lineair} zijn. Hier kun je aan fitten:
  \begin{equation*}
    y = A + Bx + Cx^2 + Dx^3 + \ldots
  \end{equation*}
  \pause
  En ook:
  \begin{equation*}
    y = A\sin x + B\cos x
  \end{equation*}
  \pause
  Maar niet:
  \begin{equation*}
    y = A\cdot e^{Bx}
  \end{equation*}
  \pause
  Daar staat tegenover dat je wel weer kunt \alert{lineariseren}:
  \begin{equation*}
    \ln y = \ln A + Bx  \qquad (y' = A' + Bx)
  \end{equation*}
\end{frame}

\begin{frame}{De computer kan alles}
  De computer is goed in dom rekenwerk, en kan verschillende waardes voor $A$ en $B$ blijven proberen, net zolang totdat de \emph{som van kwadraten} $h$ klein genoeg is! Dus óók functies die niet te fitten zijn met lineaire regressie (bv. $y = A\cdot e^{Bx}$) kun je zo alsnog fitten.
\end{frame}

\end{document}
